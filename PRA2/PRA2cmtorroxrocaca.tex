% Options for packages loaded elsewhere
\PassOptionsToPackage{unicode}{hyperref}
\PassOptionsToPackage{hyphens}{url}
%
\documentclass[
]{article}
\usepackage{amsmath,amssymb}
\usepackage{lmodern}
\usepackage{iftex}
\ifPDFTeX
  \usepackage[T1]{fontenc}
  \usepackage[utf8]{inputenc}
  \usepackage{textcomp} % provide euro and other symbols
\else % if luatex or xetex
  \usepackage{unicode-math}
  \defaultfontfeatures{Scale=MatchLowercase}
  \defaultfontfeatures[\rmfamily]{Ligatures=TeX,Scale=1}
\fi
% Use upquote if available, for straight quotes in verbatim environments
\IfFileExists{upquote.sty}{\usepackage{upquote}}{}
\IfFileExists{microtype.sty}{% use microtype if available
  \usepackage[]{microtype}
  \UseMicrotypeSet[protrusion]{basicmath} % disable protrusion for tt fonts
}{}
\makeatletter
\@ifundefined{KOMAClassName}{% if non-KOMA class
  \IfFileExists{parskip.sty}{%
    \usepackage{parskip}
  }{% else
    \setlength{\parindent}{0pt}
    \setlength{\parskip}{6pt plus 2pt minus 1pt}}
}{% if KOMA class
  \KOMAoptions{parskip=half}}
\makeatother
\usepackage{xcolor}
\usepackage[margin=1in]{geometry}
\usepackage{graphicx}
\makeatletter
\def\maxwidth{\ifdim\Gin@nat@width>\linewidth\linewidth\else\Gin@nat@width\fi}
\def\maxheight{\ifdim\Gin@nat@height>\textheight\textheight\else\Gin@nat@height\fi}
\makeatother
% Scale images if necessary, so that they will not overflow the page
% margins by default, and it is still possible to overwrite the defaults
% using explicit options in \includegraphics[width, height, ...]{}
\setkeys{Gin}{width=\maxwidth,height=\maxheight,keepaspectratio}
% Set default figure placement to htbp
\makeatletter
\def\fps@figure{htbp}
\makeatother
\setlength{\emergencystretch}{3em} % prevent overfull lines
\providecommand{\tightlist}{%
  \setlength{\itemsep}{0pt}\setlength{\parskip}{0pt}}
\setcounter{secnumdepth}{-\maxdimen} % remove section numbering
\usepackage{booktabs}
\usepackage{longtable}
\usepackage{array}
\usepackage{multirow}
\usepackage{wrapfig}
\usepackage{float}
\usepackage{colortbl}
\usepackage{pdflscape}
\usepackage{tabu}
\usepackage{threeparttable}
\usepackage{threeparttablex}
\usepackage[normalem]{ulem}
\usepackage{makecell}
\usepackage{xcolor}
\ifLuaTeX
  \usepackage{selnolig}  % disable illegal ligatures
\fi
\IfFileExists{bookmark.sty}{\usepackage{bookmark}}{\usepackage{hyperref}}
\IfFileExists{xurl.sty}{\usepackage{xurl}}{} % add URL line breaks if available
\urlstyle{same} % disable monospaced font for URLs
\hypersetup{
  pdftitle={Pràctica 2: Com realitzar la neteja i l'anàlisi de dades?},
  pdfauthor={Carlos Martínez Torró (cmtorro@uoc.edu) i Xavier Roca Canals (xrocaca@uoc.edu)},
  hidelinks,
  pdfcreator={LaTeX via pandoc}}

\title{\textbf{Pràctica 2: Com realitzar la neteja i l'anàlisi de
dades?}}
\author{Carlos Martínez Torró
(\href{mailto:cmtorro@uoc.edu}{\nolinkurl{cmtorro@uoc.edu}}) i Xavier
Roca Canals
(\href{mailto:xrocaca@uoc.edu}{\nolinkurl{xrocaca@uoc.edu}})}
\date{11/01/2023}

\begin{document}
\maketitle

\renewcommand*\contentsname{Índex de la pràctica}
{
\setcounter{tocdepth}{2}
\tableofcontents
}
\newpage

\hypertarget{exercici-1-descripciuxf3-del-dataset}{%
\section{Exercici 1: Descripció del
dataset}\label{exercici-1-descripciuxf3-del-dataset}}

\href{https://zenodo.org/record/7343400}{El nostre dataset} recopila
desenes de mètriques estadístiques de més de 70 temporades de la NBA i
de 25 anys de la WNBA, és a dir, des dels inicis respectius de cada
lliga. Les dades es van recollir del web
\href{https://www.basketball-reference.com/}{Basketball Reference}
(propietat del grup Sports Reference) i estan classificades per equip i
per temporada, amb un total de 1900 observacions i 53 variables. És a
dir, a cada fila trobarem com li ha anat a un equip, des del punt de
vista mètric, en una temporada en concret. L'ampla disponibilitat de
variables ens permet anar des dels anàlisis més superficials (victòries,
derrotes, punts a favor\ldots) a altres més complexos (ritme de joc,
eficiència dels llançaments\ldots).

En una era com la nostra en la que les dades s'han infiltrat arreu, ens
preguntem si també han arribat a la lliga de bàsquet per excel·lència.
L'objectiu que perseguim amb la creació i l'actual anàlisi d'aquest
dataset és ambiciós: es pot explicar l'evolució de la NBA a través de
les estadístiques? Hi ha algun patró que segueixin aquells equips més
exitosos? I estan dirigint aquests equips l'evolució de l'esport?

\hypertarget{exercici-2-integraciuxf3-i-selecciuxf3-de-les-dades}{%
\section{Exercici 2: Integració i selecció de les
dades}\label{exercici-2-integraciuxf3-i-selecciuxf3-de-les-dades}}

A la Pràctica 1 ja vam realitzar una integració de diferents datasets
creats a partir del \emph{web scraping}, ja que estàvem interessats en
diferents taules i vam optar per fer un dataframe per cadascuna d'elles
i després unir els dataframes resultants per les columnes comunes (amb
la funció \texttt{merge} de la llibreria \texttt{pandas}). Per tant, el
dataset lliurat a la PRA1 és el que carregarem.

Com posar la sortida de les funcions \texttt{head}, \texttt{summary} o
\texttt{str} pot allargar molt el document i resultar improductiu degut
a l'elevat nombre de variables que tenim, podem crear una taula per
veure la informació bàsica del dataset: quantes observacions tenim,
quantes variables, quantes són numèriques, quantes categòriques, etc.

\begin{table}[!h]

\caption{\label{tab:unnamed-chunk-2}Mètriques bàsiques del dataset}
\centering
\begin{tabular}[t]{ll}
\toprule
Mètrica & Valor\\
\midrule
\cellcolor{gray!6}{Nombre d'observacions} & \cellcolor{gray!6}{1900}\\
Nombre de variables & 53\\
\cellcolor{gray!6}{Variables numèriques} & \cellcolor{gray!6}{48}\\
Variables categòriques & 5\\
\cellcolor{gray!6}{Casos complets (\%) (observacions sense NAs)} & \cellcolor{gray!6}{1522 / 1900  =  80.11 \%}\\
\addlinespace
Variables completes (\%) (variables sense NAs) & 31 / 53  =  58.49 \%\\
\bottomrule
\end{tabular}
\end{table}

Com podem veure a la taula, hi ha un gran percentatge de casos complets
(és a dir, equips amb tota la informació de l temporada completa),
mentre que aproximadament el 60\% de les variables no tenen cap valor
NA. Podem fer una ullada a quines són les variables amb més valors
perduts del nostre dataset.

\begin{table}[!h]

\caption{\label{tab:unnamed-chunk-3}Variables amb més valors perduts}
\centering
\begin{tabular}[t]{lr}
\toprule
Variable & Valors NA\\
\midrule
\cellcolor{gray!6}{fg3a\_per\_fga\_pct} & \cellcolor{gray!6}{378}\\
fg3 & 378\\
\cellcolor{gray!6}{fg3a} & \cellcolor{gray!6}{378}\\
fg3\_pct & 378\\
\cellcolor{gray!6}{tov\_pct} & \cellcolor{gray!6}{259}\\
\addlinespace
orb\_pct & 259\\
\cellcolor{gray!6}{opp\_tov\_pct} & \cellcolor{gray!6}{259}\\
drb\_pct & 259\\
\cellcolor{gray!6}{orb} & \cellcolor{gray!6}{259}\\
drb & 259\\
\bottomrule
\end{tabular}
\end{table}

Veiem que algunes d'aquestes variables estan relacionades amb el
llançament de tres punts. Això és degut a que fins 1979 no existia
aquest llançament, així que hi ha gairebé trenta anys de registres on
aquestes variables tenen valor nul per definició. Per altra banda,
algunes mètriques com els rebots ofensius (variable \emph{orb}) o
defensius (variable \emph{drb}) i les pèrdues (com el percentatge del
propi equip amb la variable \emph{tov\_pct} o del contrari amb
\emph{opp\_tov\_pct}) també van ser estadístiques que no es recollien
originalment, així que és normal que hi hagi valors perduts en aquestes.
De fet, si filtrem i només agafem les observacions posteriors a la
temporada 1978-79 (el llançament de tres va començar a la 1979-80)
veurem que el nombre de valors perduts és molt diferent:

\begin{verbatim}
## Valors perduts després de la temporada 1978-79: 0
\end{verbatim}

\hypertarget{exercici-3-neteja-de-les-dades}{%
\section{Exercici 3: Neteja de les
dades}\label{exercici-3-neteja-de-les-dades}}

Com bé hem comentat en l'anterior apartat, trobem temporades en les
quals ens falta informació sobre estadístiques bàsiques sobre el joc.
D'aquesta forma, ens dificulta l'anàlisi sobre aquestes temporades, ja
que no disposem dels elements bàsics per entendre com era l'estil o
funcionament de joc de cada equip. En conseqüència, s'ha decidit
descartar aquestes temporades.

\hypertarget{les-dades-contenen-zeros-o-elements-buits}{%
\subsection{3.1: Les dades contenen zeros o elements
buits?}\label{les-dades-contenen-zeros-o-elements-buits}}

Així i tot, trobem temporades més antigues a l'aparició del tir de tres
punts que si disposen d'aquestes estadístiques. Observant el conjunt de
dades, trobem que a partir de la temporada 1973-74 disposem de tota la
informació necessària. Així i tot, analitzarem quins valors buits
disposem en el nostre conjunt de dades a partir d'aquesta temporada.

\begin{verbatim}
## fg3a_per_fga_pct              fg3             fg3a          fg3_pct 
##              102              102              102              102 
##           Season           League             Team             wins 
##                0                0                0                0 
##           losses     win_loss_pct               gb        pts_per_g 
##                0                0                0                0 
##    opp_pts_per_g         Playoffs              age        wins_pyth 
##                0                0                0                0 
##      losses_pyth              mov              sos              srs 
##                0                0                0                0 
##          off_rtg          def_rtg          net_rtg             pace 
##                0                0                0                0 
##  fta_per_fga_pct           ts_pct          efg_pct          tov_pct 
##                0                0                0                0 
##          orb_pct          ft_rate      opp_efg_pct      opp_tov_pct 
##                0                0                0                0 
##          drb_pct      opp_ft_rate                g               mp 
##                0                0                0                0 
##               fg              fga           fg_pct              fg2 
##                0                0                0                0 
##             fg2a          fg2_pct               ft              fta 
##                0                0                0                0 
##           ft_pct              orb              drb              trb 
##                0                0                0                0 
##              ast              stl              blk              tov 
##                0                0                0                0 
##               pf 
##                0
\end{verbatim}

Com era d'esperar, les úniques variables que observem amb valors buits
són les que fan referència als tirs de tres punts. El que farem, és
omplir aquests valors buits amb el valor 0, ja que s'ha decidit que és
el valor que realment representa aquests camps. Com no existia el tir de
tres no es van realitzar cap tir.

Un altre fet a tenir en compte, és el camp \texttt{gb}. Aquesta variable
ens mostra quants partits té cada equip per darrere del rival que ocupa
el primer lloc. En alguns casos, aquesta informació ve representada amb
el valor \texttt{-}. Entenem que aquest valor, fa referència al fet que
no té cap partit per darrere dels rivals (és a dir, és l'equip que va
primer de la seva divisió) i el seu valor real és 0.

\hypertarget{identifica-i-gestiona-els-valors-extrems.}{%
\subsection{3.2: Identifica i gestiona els valors
extrems.}\label{identifica-i-gestiona-els-valors-extrems.}}

En aquest cas, a partir de la funció `summary' podem observar els mínims
i màxims valors de cada variable. No s'exemplificarà en el document, ja
que el gran volum de variables dificultaria la lectura del document de
la pràctica. Així i tot, un cop comprovat aquest, sí que podem trobar
valors molt petits en comparació a la mitjana, com pot ser el cas de
victòries que ha assolit un equip en una temporada.

Ho podem veure en aquest petit exemple:

\begin{verbatim}
##    Min. 1st Qu.  Median    Mean 3rd Qu.    Max. 
##    2.00   22.00   36.00   35.41   47.00   73.00
\end{verbatim}

És normal, que ens podem trobar aquests casos en múltiples variables, ja
que en una mateixa temporada, els equips guanyadors presentaran valors
màxims en victòries i els perdedors mínims en aquestes. En conseqüència,
també es veurà reflectit en les estadístiques d'aquests equips.

El que s'ha decidit és no realitzar cap modificació, ja que aquests
valors són correctes perquè es basen en dades reals sobre les temporades
i les estadístiques de joc de cada un dels equips. Per tant, hem
d'assumir que serà possible trobar valors que siguin \emph{outliers},
però que haurem de tractar com un valor més.

\hypertarget{exercici-4-anuxe0lisi-de-les-dades}{%
\section{Exercici 4: Anàlisi de les
dades}\label{exercici-4-anuxe0lisi-de-les-dades}}

Abans de començar amb l'anàlisi, haurem de tenir en compte que els
valors absoluts ens poden portar a error i que haurem d'optar, en
general, pels relatius (percentatges o mètriques ajustades per partit o
possessions). Això és degut a que la NBA i la WNBA tenen un nombre de
partits totals diferents, però també perquè la pròpia NBA ha evolucionat
en aquest aspecte: en el seu inici es jugaven molts menys partits.
Veiem-ho amb un gràfic, on utilitzarem les dades de partits des de 1973:

\includegraphics{PRA2cmtorroxrocaca_files/figure-latex/unnamed-chunk-9-1.pdf}

Com es pot veure, en general les temporades a la NBA han tingut més de
80 partits, però no sempre ha sigut així. Pel que fa a la WNBA, hi ha
més variació en aquest número de partits, però es troba al voltant dels
30. És per això que, en cas d'utilitzar valors absoluts, haurem de
comprovar prèviament que estiguem comparant entre temporades amb el
mateix número de partits.

\hypertarget{selecciuxf3-dels-grups-de-dades-que-es-volen-analitzarcomparar.}{%
\subsection{4.1: Selecció dels grups de dades que es volen
analitzar/comparar.}\label{selecciuxf3-dels-grups-de-dades-que-es-volen-analitzarcomparar.}}

Per seleccionar els grups de dades a comparar, farem èmfasis en els
mètodes d'anàlisi que es volen realitzar per tal d'escollir quines
variables/camps ens poden ajudar per donar resposta a aquests.

\begin{itemize}
\item
  \textbf{Anàlisi estadística descriptiva}: Realitzarem una exploració
  de les dades per tal de veure quins equips presenten una mitjana de \%
  de victòries més elevada durant els anys. Així doncs, ho compararem
  amb el percentatge de les vegades que ha estat classificat cada equip
  a playoffs per visualitzar la relació que poden tenir. A més, veurem
  l'evolució d'estadístiques bàsiques de cada temporada per veure la
  seva evolució i entendre quina tendència segueixen per avaluar com ha
  pogut canviar l'estil de joc de les grans lligues durant els anys.
\item
  \textbf{Anàlisi estadística inferencial}: En aquest cas, realitzarem
  diferents regressions entre les estadístiques bàsiques en comparació a
  les victòries per tal d'analitzar la importància d'aquestes en el
  rendiment dels equips.
\item
  \textbf{Model supervisat}: Finalment, realitzarem un model predictor
  que ens identifiqui si un equip té la possibilitat d'arribar a
  playoffs segons certes variables més avançades que intervenen en el
  joc.
\end{itemize}

Per tant, el conjunt de dades final tindria les següents variables:

\textbf{Faltaria ficar la taula. Pendent del que vulguis fer}

\hypertarget{comprovaciuxf3-de-la-normalitat-i-homogeneuxeftat-de-la-variuxe0ncia.}{%
\subsection{4.2: Comprovació de la normalitat i homogeneïtat de la
variància.}\label{comprovaciuxf3-de-la-normalitat-i-homogeneuxeftat-de-la-variuxe0ncia.}}

Per tal de realitzar la comprovació de la normalitat, ens basarem en el
test de Shapiro-Wilk, ja que es tracta dels més potents per contrastar
aquesta. Assumirem com a hipòtesi nul·la que la població està
distribuïda normalment. Direm que \(\alpha\) = 0.05, per tant, si
p-valor és major a \(\alpha\), assumirem que les dades segueixen una
distribució normal.

\begin{verbatim}
## 
##  Shapiro-Wilk normality test
## 
## data:  df3$fg3
## W = 0.9643, p-value < 2.2e-16
\end{verbatim}

Seguidament, per comprovar l'homogeneïtat de la variància, ens basarem
en els tests de Levene, per les que segueixen una distribució normal i
Fligner-Killeen, per les que no les compleixen. Assumirem com a hipòtesi
nul·la la igualtat de variàncies en els grups de dades. Direm que
\(\alpha\) = 0.05, per tant, si p-valor és major a \(\alpha\), no podrem
refutar aquesta hipòtesi nul·la i, per tant, podrem dir que hi haurà
igualtat de variàncies.

\begin{verbatim}
## 
##  Fligner-Killeen test of homogeneity of variances
## 
## data:  wins by fg3
## Fligner-Killeen:med chi-squared = 194.45, df = 151, p-value = 0.009865
\end{verbatim}

\hypertarget{aplicaciuxf3-de-proves-estaduxedstiques-per-comparar-els-grups-de-dades.-aplicar-almenys-tres-muxe8todes-danuxe0lisi-diferents.}{%
\subsection{4.3: Aplicació de proves estadístiques per comparar els
grups de dades. Aplicar almenys tres mètodes d'anàlisi
diferents.}\label{aplicaciuxf3-de-proves-estaduxedstiques-per-comparar-els-grups-de-dades.-aplicar-almenys-tres-muxe8todes-danuxe0lisi-diferents.}}

Seguidament, aplicarem les proves estadístiques descrites anteriorment.

\hypertarget{anuxe0lisi-estaduxedstica-descriptiva}{%
\subsubsection{Anàlisi estadística
descriptiva}\label{anuxe0lisi-estaduxedstica-descriptiva}}

Començarem agrupant el percentatge de victòries i percentatge de vegades
que ha entrat a playoffs cada equip i realitzarem la mitjana a partir
del nombre de temporades. D'aquesta forma, estudiarem quins equips són
els equips més guanyadors durant els anys i com es veu influenciat en
les classificacions als playoffs.
\includegraphics{PRA2cmtorroxrocaca_files/figure-latex/unnamed-chunk-12-1.pdf}

Com es pot observar, veiem una alta relació en la mitjana de victòries
aconseguides amb el percentatge de vegades que ha entrat l'equip a
playoffs. Aquest fet pot resultar evident, ja que amb més victòries
obtingudes en una temporada més possibilitats d'entrar a playoffs. Així
i tot, si veiem a l'NBA, l'equip TOP 5 (Oklahoma City Thunder) té un
percentatge proper a 60\% de victòries i supera un 70\% de vegades que
ha classificat a playoffs. Per altra banda, si ho comparem amb el top 5
de la WNBA (Minnesota Lynx). Aquest té un percentatge de victòries
proper al 50\% i el seu \% de vegades que ha entrat a playoffs és de
quasi un 50\% també. Fet que ens podria assegurar que aquest equip pot
resultar irregular (diferència elevada de victòries/derrotes en les
temporades) o que frega els límits de classificació cada temporada, fet
que desemboca en aquesta irregularitat d'arribada a playoffs.

\hypertarget{anuxe0lisi-1-ha-canviat-la-preferuxe8ncia-dels-llanuxe7aments}{%
\subsection{Anàlisi 1: ha canviat la preferència dels
llançaments?}\label{anuxe0lisi-1-ha-canviat-la-preferuxe8ncia-dels-llanuxe7aments}}

La introducció dels llançaments de tres a la lliga l'any 1979 va suposar
un abans i un després a l'hora de jugar l'esport. No obstant, un
espectador que no hagi vist res de bàsquet en els darrers 20 anys es
podria sorprendre amb la quantitat aparent de llançaments de tres que es
practiquen avui en dia. Amb l'aparició de fenòmens com Stephen Curry, el
joc sembla haver canviat en els darrers anys.

\hypertarget{anuxe0lisi-1.1-selecciuxf3-de-les-dades-que-es-volen-analitzarcomparar}{%
\subsubsection{Anàlisi 1.1: Selecció de les dades que es volen
analitzar/comparar}\label{anuxe0lisi-1.1-selecciuxf3-de-les-dades-que-es-volen-analitzarcomparar}}

Per analitzar-ho millor, farem un anàlisi dècada per dècada dels
llançaments de tres. Ens fixarem en una variable en concret:
\texttt{fg3a\_per\_fga\_pct}. Aquesta variable indica (en percentatge)
quants llançaments de tres ha realitzat un equip de tots els llançaments
intentats. És a dir, si de 10 tirs de camp, 4 són triples, parlarem d'un
40\% (en la variable estaria codificat com a 0.4). Triem aquesta
variable en comptes del nombre de triples perquè ens pot donar una idea
millor de si ha variat la selecció de llançaments.

Així doncs, farem quatre grups diferents: del 79 fins el 90, del 90 fins
el 2000, del 2000 fins el 2010 i del 2010 fins l'actualitat. Ho
codificarem tot en una nova variable, que anomenarem \texttt{decade}:

Amb un boxplot podem comparar les èpoques a primera vista:

\includegraphics{PRA2cmtorroxrocaca_files/figure-latex/unnamed-chunk-14-1.pdf}

Veiem que, efectivament, hi ha una tendència a llançar més de tres amb
els anys. Utilitzem ara un gràfic de dispersió per a observar aquesta
tendència amb el pas de les temporades:

\includegraphics{PRA2cmtorroxrocaca_files/figure-latex/unnamed-chunk-15-1.pdf}

Notablement, sembla que aquest percentatge es manté en un rang similar
des de mitjans dels 90 fins mitjans de la dècada dels 2010s, on comença
a pujar. Hi ha un equip, fins i tot, que va sobrepassar el 50\% de
llançaments de tres del total de llançaments.

\begin{table}[!h]

\caption{\label{tab:unnamed-chunk-16}Equip amb el valor màxim de llançaments de tres intentats 
      respecte el total de llançaments}
\centering
\begin{tabular}[t]{llrrlrr}
\toprule
Season & Team & wins & losses & Playoffs & fg3a\_per\_fga\_pct & fg3\_pct\\
\midrule
\cellcolor{gray!6}{2018-19} & \cellcolor{gray!6}{Houston Rockets} & \cellcolor{gray!6}{53} & \cellcolor{gray!6}{29} & \cellcolor{gray!6}{Yes} & \cellcolor{gray!6}{0.519} & \cellcolor{gray!6}{0.356}\\
\bottomrule
\end{tabular}
\end{table}

\hypertarget{anuxe0lisi-1.2-comprovaciuxf3-de-la-normalitat-i-homogeneuxeftat-de-la-variuxe0ncia}{%
\subsubsection{Anàlisi 1.2: Comprovació de la normalitat i homogeneïtat
de la
variància}\label{anuxe0lisi-1.2-comprovaciuxf3-de-la-normalitat-i-homogeneuxeftat-de-la-variuxe0ncia}}

Passarem ara a fer l'anàlisi estadístic d'aquestes dades. Abans, però,
haurem de comprovar la normalitat per saber si haurem d'aplicar un test
paramètric o un no paramètric.

\begin{table}[!h]

\caption{\label{tab:unnamed-chunk-17}Test de Shapiro-Wilk per avaluar normalitat de la mostra}
\centering
\begin{tabular}[t]{lrr}
\toprule
decade & statistic & p.value\\
\midrule
\cellcolor{gray!6}{80s} & \cellcolor{gray!6}{0.8818854} & \cellcolor{gray!6}{0.0000000}\\
90s & 0.9804219 & 0.0003230\\
\cellcolor{gray!6}{00s} & \cellcolor{gray!6}{0.9900004} & \cellcolor{gray!6}{0.0045421}\\
10s & 0.9875335 & 0.0001305\\
\bottomrule
\end{tabular}
\end{table}

Com es pot veure a la taula, el p-valor és en tots els casos molt
inferior a 0.05, el que permet refutar la hipòtesi nul·la que assumeix
una distribució normal. Per tant, podem dir que cap de les quatre
mostres avaluades té una distribució normal per la variable
\texttt{fg3a\_per\_fga\_pct}.

Pel que fa a la variància, podem comparar les variàncies amb el test de
Bartlett o el test de Levene.

\begin{table}[!h]

\caption{\label{tab:unnamed-chunk-18}Avaluació de l'homoscedasticitat amb els tests de Bartlett i Levene}
\centering
\begin{tabular}[t]{lr}
\toprule
Test & p.valor\\
\midrule
\cellcolor{gray!6}{Test de Bartlett} & \cellcolor{gray!6}{0}\\
Test de Levene & 0\\
\bottomrule
\end{tabular}
\end{table}

Podem comprovar mirant els p-valors d'ambdós tests que les mostres no
tenen la mateixa variància, ja que en ambdós casos refutem la hipòtesi
nul·la de igualtat de variàncies.

\hypertarget{anuxe0lisi-1.3-aplicaciuxf3-de-proves-estaduxedstiques-per-comparar-els-grups-de-dades.}{%
\subsubsection{Anàlisi 1.3: Aplicació de proves estadístiques per
comparar els grups de
dades.}\label{anuxe0lisi-1.3-aplicaciuxf3-de-proves-estaduxedstiques-per-comparar-els-grups-de-dades.}}

Per tant, tenim mostres on no hi ha una distribució normal dels valors i
tampoc tenim una situació de homoscedasticitat. No obstant, com tenim
una mostra força gran, podem aplicar el \textbf{teorema del límit
central}, ja que, degut a la mida de la mostra, podem assumir que si fem
les mitjanes aritmètiques de diferents mostrejos aleatoris, la
distribució d'aquestes mitjanes aritmètiques serà gaussiana.

\begin{table}[!h]

\caption{\label{tab:unnamed-chunk-19}Nombre d'observacions que tenim per dècada}
\centering
\begin{tabular}[t]{lr}
\toprule
Dècada & Observacions\\
\midrule
\cellcolor{gray!6}{Pre-three era} & \cellcolor{gray!6}{378}\\
80s & 231\\
\cellcolor{gray!6}{90s} & \cellcolor{gray!6}{308}\\
00s & 437\\
\cellcolor{gray!6}{10s} & \cellcolor{gray!6}{546}\\
\bottomrule
\end{tabular}
\end{table}

Pel que fa a la variància, podem aplicar el test de Welch, que és una
alternativa a l'ANOVA clàssic quan no hi ha homoscedasticitat. Així
doncs, mirarem si hi ha diferències entre les diferents dècades pel que
fa al percentatge de llançaments de tres respecte el total de
llançaments fets.

\begin{table}[!h]

\caption{\label{tab:unnamed-chunk-20}Test de Welch per avaluar diferència entre els llançaments de tres respecte el total per cada dècada}
\centering
\begin{tabular}[t]{lr}
\toprule
Test & p.valor\\
\midrule
\cellcolor{gray!6}{ANOVA de Welch} & \cellcolor{gray!6}{0}\\
\bottomrule
\end{tabular}
\end{table}

El p-valor de 0 ens indica que hi ha diferències. Ara bé, entre quins
grups? Per saber-ho, necessitem fer un test \emph{post hoc}. Podem
utilitzar el test de Games-Howell, similar al test de Tukey (un dels més
comuns), però aquest no assumeix igualtat de variàncies.

\begin{table}[!h]

\caption{\label{tab:unnamed-chunk-21}Diferències entre dècades en la variable fg3a_per_fga_pct}
\centering
\begin{tabular}[t]{lll}
\toprule
Grup 1 & Grup 2 & p-Valor ajustat\\
\midrule
\cellcolor{gray!6}{80s} & \cellcolor{gray!6}{90s} & \cellcolor{gray!6}{0}\\
80s & 00s & 0\\
\cellcolor{gray!6}{80s} & \cellcolor{gray!6}{10s} & \cellcolor{gray!6}{0}\\
90s & 00s & 4.9e-10\\
\cellcolor{gray!6}{90s} & \cellcolor{gray!6}{10s} & \cellcolor{gray!6}{0}\\
\addlinespace
00s & 10s & 1.98e-13\\
\bottomrule
\end{tabular}
\end{table}

Així, veiem que hi ha diferències significatives entre tots els grups.
No obstant, veiem que les diferències entre els 90s i els 2000s i entre
els 2000s i els 2010s són menors que entre les altres dècades, un fet
que podíem intuir amb la representació gràfica d'aquesta variable.

\hypertarget{exercici-5-representaciuxf3-dels-resultats.}{%
\section{Exercici 5: Representació dels
resultats.}\label{exercici-5-representaciuxf3-dels-resultats.}}

\hypertarget{exercici-6-resoluciuxf3-del-problema.}{%
\section{Exercici 6: Resolució del
problema.}\label{exercici-6-resoluciuxf3-del-problema.}}

\hypertarget{exercici-7-codi.}{%
\section{Exercici 7: Codi.}\label{exercici-7-codi.}}

\hypertarget{exercici-8-vuxeddeo.}{%
\section{Exercici 8: Vídeo.}\label{exercici-8-vuxeddeo.}}

\end{document}
